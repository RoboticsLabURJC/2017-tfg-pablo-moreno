\chapter{Conclusiones}\label{cap.conclusiones}
Una vez documentadas en profundidad las dos prácticas desarrolladas en este Trabajo de Fin de Grado, se dedica este capítulo a la comprobación de los objetivos alcanzados, así como a la explicación de los conocimientos adquiridos y una breve exposición de posibles mejoras de las prácticas y de las líneas de actuación futuras.

\section{Conclusiones}
El objetivo principal de desarrollar una nueva práctica y la actualización y mejora notoria en otra de las prácticas del entorno \textit{Robotics-Academy} ha sido alcanzado con éxito. Este objetivo principal estaba dividido en dos subobjetivos.

El primer subobjetivo fue el desarrollo de una práctica completamente nueva llamada \textit{Chrono}. Este ejercicio fue bastante complejo debido a la implementación de una reproducción sincronizada del coche Fórmula 1 de referencia que se adapte al rendimiento de cada ordenador. Además, la práctica debe grabar la simulación actual para su posterior reproducción en el caso de que el algoritmo desarrollado sea más rápido que la grabación de referencia. Adicionalmente, otro punto de gran complejidad era la visión de la posición del F1 con el récord del circuito en la interfaz gráfica del nodo ROS, punto que también se consiguió solventar. A parte de estas tareas especialmente complejas, se desarrolló la práctica desde cero, incluyendo el nodo académico, la conexión con los sensores actuadores del robot y el fichero para albergar la solución del estudiante.

El segundo subobjetivo establecido fue la mejora de la práctica sigue carreteras o \textit{Follow Road} con dron, existente en el entorno de Robotics-Academy. Este objetivo se logró alcanzar, dado que la infraestructura de la práctica fue completamente renovada y mejorada, incluyendo \textit{drivers} de ROS para dar un mejor soporte a su infraestructura, integrando \textit{Px4, MAVLink y MavROS} para drones. Además, se desarrollaron mejoras en la interfaz gráfica y el nodo académico, la introducción de un visor de imágenes filtradas. Se tuvo que hacer una readaptación de las conexiones del nodo académico con sensores y actuadores del robot. Después de todas estas modificaciones se consiguió una práctica totalmente operativa y actualizada. Tanto es así, que la renovación de la interfaz gráfica y la adaptación de \textit{ROS} para drones se están incluyendo en otras prácticas similares del entorno Robotics-Academy que también manejan drones.

En relación con el desarrollo de una solución de referencia para cada práctica. Este punto incluía el estudio de técnicas de adquisición, procesado y control de imágenes y control del movimiento del robot. Ambas soluciones se programaron satisfactoriamente compartiendo mucho código de procesado de imágenes.

Para la solución de la primera práctica se tuvo que realizar un estudio paralelo de filtrado y postprocesado de imagen para captar y procesar las imágenes grabadas por la cámara del coche. Además de adquirir los conocimientos de movimiento para el robot F1 utilizando los \textit{drivers} de ROS. Es importante destacar que en este caso se tuvo que realizar una solución más rápida que el algoritmo con el récord del circuito optimizado a partir de una solución previa. Es decir, en este caso se tuvieron que realizar dos soluciones distintas, una optimizada y otra desde cero.

Para la solución de la segunda práctica fue necesario el estudio de técnicas de procesado de imágenes para realizar un filtro de color para filtrar la carretera del resto del escenario. Una vez realizado el primer filtro fue necesario aplicar técnicas del postprocesado de imágenes para adaptar la imagen filtrada y localizar el centro de la carretera. Además, fue necesario el estudio de movimiento del dron para realizar un movimiento controlado. Este movimiento es muy sensible debido al control de la actitud y la inclinación del dron que pueden afectar a la visualización de las imágenes captadas por la cámara.

Finalmente, se realizó una readaptación de la práctica \textit{Chrono} y \textit{Follow\_Road} para su inclusión en la plataforma web Jupyter. De esta manera el estudiante dispondrá de las mismas prácticas que las disponibles en el entorno \textit{Robotics-Academy} pero con cuadernillos en lugar de nodo ROS. De esta manera, el nodo académico es transparente al alumno, que dispone de una celda para desarrollar su algoritmo y solo tendrá que ejecutar dicha celda para ver su estado.
Para lograr dicho objetivo fue necesario adquirir conocimientos de Jupyter así como un estudio de nuevos procesos de Python.

Por último, gracias a una motivación personal, se han adquirido, durante este TFG, conocimientos para afrontar problemas reales de ingeniería en el campo de desarrollo software. Se ha adquirido experiencia para integrarse en proyectos de robótica, desde su infraestructura, las conexiones hardware-software, las interfaces, componentes, funcionalidad, así como la parte visible al usuario. También se han adquiridos conocimientos de simulación, y diseño gráfico, así como herramientas de tratamiento de imagen y técnicas de programación.

\section{Trabajos futuros} 
Este TFG ofrece dos ejercicios totalmente renovados a la plataforma educativa \textit{Robotics-Academy}. Éstos son susceptibles de mejora y además se pueden extender con otros ejercicios que aprovechen la infraestructura.

Relacionado con la práctica de \textit{Chrono}, sería interesante el desarrollo de una competición en distintos circuitos y la inclusión de distintos coches, cada uno con su propio algoritmo, que compitan entre ellos para conocer el algoritmo más eficiente en cada circuito. De esta manera se estaría desarrollando un Grand Pix de robots F1.

Relacionado con la práctica del dron que ha de seguir una carretera, una posibilidad es realizar un ejercicio \textit{Delivery-Drone}, es decir, un dron con inteligencia parecida a la de esa misma práctica. De esta manera podría realizarse una práctica en la que, con un mapa previo, se le indique al dron al punto al que debe dirigirse en el plano de una ciudad. De esta manera se podría dotar al dron de la inteligencia necesaria para ser un ``Dron repartidor'' utilizando algoritmos de planificación.

Otro punto de desarrollo futuro podría ser la realización de nuevos algoritmos más eficientes y rápidos que las soluciones de referencia propuestas en este Trabajo de Fin de Grado.
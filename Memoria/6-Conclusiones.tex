\chapter{Conclusiones}\label{cap.conclusiones}
Una vez documentadas en profundidad las dos prácticas desarrolladas en este Trabajo de Fin de Grado, se dedicará el siguiente capítulo a la comprobación de la consecución de los objetivos alcanzados, así como a la explicación de los conocimientos adquiridos y una breve exposición de posibles mejoras de las prácticas y de las líneas de actuación futuras.

\section{Conclusiones}
El objetivo principal de desarrollar una nueva práctica y la consecución de una mejora notoria en otra de las prácticas del entorno JdeRbot-Academy ha sido alcanzado con éxito. Prueba de ello son, tanto su inclusión en el programa de prácticas, como el correcto funcionamiento del nodo académico y de la solución de referencia desarrollada. Este objetivo principal estaba subdividido en objetivos secundarios de los cuales se detallará a continuación si han sido alcanzado y la manera para ello.

El primer objetivo secundario establecido fue la mejora de una práctica existente del entorno de JdeRobot-Academy. La práctica escogida fue el sigue carreteras o \textit{Follow Road}. Este objetivo se logró alcanzar dado que la práctica fue completamente optimizada y mejorada. Se desarrollaron mejoras en la interfaz gráfica y el nodo académico, como la introducción de un visor de imágenes filtradas y el desarrollo de una pausa académica. Además, se realizó una optimización de la práctica incluyendo \textit{drivers} de ROS para der un mejor soporte a su infraestructura. De esta manera se tuvo que hacer una readaptación de las conexiones del nodo académico con sensores y actuadores del robot. Después de todas estas modificaciones se consiguió una práctica totalmente operativa y actualizada. Tanto es así, que la renovación de la interfaz gráfica se incluyó en las prácticas similares del entorno JdeRobot-Academy.

El siguiente objetivo fue el desarrollo de una práctica completamente nueva llamada \textit{Chrono}. Este objetivo fue bastante complejo debido a la implementación de una reproducción sincronizada que se adapte al rendimiento de cada ordenador. Además, la práctica debe grabar la simulación actual para su posterior reproducción en el caso de que el algoritmo desarrollado sea más rápido que la grabación. Adicionalmente, otro punto de gran complejidad era la visión de la posición del F1 con el récord del circuito en la interfaz gráfica del nodo académico, punto que también se consiguió solventar. A parte de estas tareas especialmente complejas, se tuvo que desarrollar la práctica desde cero, incluyendo el nodo académico, la conexión con los sensores actuadores del robot y el fichero para albergar la solución del estudiante.

En relación con los dos objetivos anteriores, se fijaron dos objetivos nuevos, los cuales establecían el desarrollo de una solución para cada práctica. Este punto incluía el estudio de técnicas de captación, procesado y control de imágnes y control del movimiento del robot. Además, sirve de punto de referencia para el desarrollo de la solución por los estudiantes. Ambos objetivos fueron alcanzados satisfactoriamente.

Para la solución de la primera práctica fue necesario el estudio de técnicas de procesado de imágenes para realizar un filtro de partículas para filtrar la carretera del resto del escenario. Una vez realizado el primer filtro fue necesario aplicar técnicas del postprocesado de imágenes para adaptar la imagen filtrada y localizar el centro de la carretera. Además, fue necesario el estudio de movimiento del dron para poder realizar un movimiento controlado. Este movimiento es muy sensible debido al control de la actitud y la inclinación del dron que pueden afectar a la visualización de las imágenes captadas por la cámara.

Para la solución de la segunda práctica se tuvo que realizar un estudio paralelo de filtrado y postprocesado de imagen para captar y procesar las imágenes grabadas por la cámara del coche. Además de adquirir los conocimientos de movimiento para el robot F1 utilizando los \textit{drivers} de ROS. Es importante destacar que en este caso se tuvo que realizar una solución más rápida que el algoritmo con el récord del circuito optimizado a partir de una solución previa. Es decir, en este caso se tuvieron que realizar dos soluciones distintas, una optimizada y otra desde cero.

Tras la consecución de los objetivos anteriores, se realizó una readaptación de las prácticas para su inclusión en la plataforma web Jupyter. De esta manera el estudiante dispondrá de las mismas prácticas que las disponibles en el entorno JdeRobot pero con cuadernillos en lugar de nodo académico. De esta manera, el nodo académico es transparente al alumno, que dispone de una celda para desarrollar su algoritmo y solo tendrá que ejecutar dicha celda para ver su estado.
Para lograr dicho objetivo fue necesario adquirir conocimientos de Jupyter así como un estudio de nuevos procesos de Python.

El siguiente objetivo respecto a la primera práctica fue su inclusión en el entorno docente JdeRobot-Academy-Web. De esta manera se da un paso más hacia un soporte multiplataforma. Esto es debido a que Academy-Web utiliza Dockers para ejecutar las prácticas en el navegador. Para poder alcanzar este objetivo hubo que estudiar el entorno JdeRobot-AcademyWeb, y el estudio de Dockers, campos totalmente desconocidos.

Por último, gracias a una motivación personal, se ha conseguido adquirir conocimientos para afrontar problemas reales de ingeniería que comprenden tanto software como hardware. Debido a esto se ha adquirido experiencia para integrarse en proyectos de robótica, desde su infraestructura, las conexiones hardware-software, las interfaces, componentes, funcionalidad, así como la part visible al usuario. También se han adquiridos conocimientos de simulación, y diseño gráfico, así como herramientas de tratamiento de imagen y técnicas de programación.

\section{Trabajos futuros} 
El presente Trabajo de Fin de Grado ha expuesto una nueva vía para la realización de proyectos en el futuro.

En la práctica del dron que ha de seguir una carretera, se expone la posibilidad de realizar un \textit{TeleTaxi}, es decir, un dron con inteligencia parecida a la de esa misma práctica. De esta manera podría realizarse una práctica en la que, con un mapa previo, se le indique al dron al punto al que debe dirigirse en un plano de ciudad. De esta manera se podría dotar al dron de la inteligencia necesaria para ser un "Dron repartidor".

Respecto a la práctica de \textit{Chrono}, sería interesante el desarrollo de una competición en distintos circuitos y la inclusión de distintos coches, cada uno con su propio algoritmo, que compitan entre ellos para conocer el algoritmo más eficiente en cada circuito. De esta manera se estaría desarrollando un Grand Pix de robots F1.

Otro punto de desarrollo futuro sería la realización de nuevos algoritmos que permitan un rendimiento más eficiente y rápido a los propuestos en este Trabajo de Fin de Grado.